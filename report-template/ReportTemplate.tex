\documentclass[a4paper,%
11pt,%
DIV=14,
%headsepline,%
headings=normal,
parskip=full
]{scrartcl}

\usepackage[utf8]{inputenc}
\usepackage[T1]{fontenc}
\usepackage[automark]{scrlayer-scrpage}
\usepackage{graphicx}
\usepackage{lmodern} 
\usepackage{url}
\usepackage{amsmath}
\usepackage{amssymb}
\usepackage{booktabs}
\usepackage{listings}
\usepackage{hyperref}


% Enable colouring
\usepackage[table]{xcolor}

%%%%%%%%%%%%%%%%%%%%%%%%%%%%%
% Set code background colour
\definecolor{light-gray}{gray}{0.95}
%%%%%%%%%%%%%%%%%%%%%%%%%%%%%

% Set up code highlighting
\usepackage{minted}
\usemintedstyle{manni}
\setminted{frame=lines, framesep=2mm, baselinestretch=1.2, fontsize=\footnotesize, linenos, bgcolor=light-gray}
\setmintedinline{frame=none, framesep=2mm, baselinestretch=0, fontsize=\footnotesize, linenos, bgcolor=white}

% Add an arrowlist
\usepackage{enumitem}
\newlist{arrowlist}{itemize}{1}
\setlist[arrowlist]{label=$\rightarrow$}

% Enable graph drawing
\usepackage{tikz}

\lstset{basicstyle=\ttfamily,frame=single}


\begin{document}
\thispagestyle{plain.scrheadings}
\noindent
\hrule height 1px
\vspace*{-1ex}
\begin{minipage}[t]{.45\linewidth}
\strut\vspace*{-\baselineskip}\newline
\hspace*{1ex}\includegraphics[height=.9cm]{./figs/Inf-Logo_black_en.pdf}
\end{minipage}
\hfill
\begin{minipage}[t]{.5\linewidth}
\flushright{
Some research group...\\%
Who am i actually working for?\\%
You can comment this out if not needed.}
\end{minipage}
\vspace*{1ex}
\hrule 
\vspace*{-1ex}
\begin{center}
{\LARGE\textbf{Some random article}}\\
{\large{}%
  \today\\
  Demo-document\\
}
\end{center}
\vspace*{2ex}
\hrule 
\vspace*{1ex}
\vspace*{-2.5ex}
1: First name, Last name, Matriculation number\\
2: First name, Last name, Matriculation number

\vspace*{1ex}
\hrule 


%%%%%%%%%%%%%%%%%%%%%%%%%%%%%%%%%%%%%%%%%%%%%%%%%%
%%%%%%          DOCUMENT GOES HERE          %%%%%%
%%%%%%%%%%%%%%%%%%%%%%%%%%%%%%%%%%%%%%%%%%%%%%%%%%

\vspace*{2ex}
This is a simple template to write reports in. It has all the features a normal \LaTeX document for this purpose should have.

\section{Supported features}
This document does support two different forms of code highlighting:\\
A simplified style that looks like this:

\begin{lstlisting}
par (0<=i<n) {
  a[i] = a[i] + b[i] * c[i];
}
\end{lstlisting}

And a fully featured one using the \texttt{minted} package:

\begin{minted}{java}

// This is a Java comment
String test = "Test!";
System.out.println(test);

if (true) {
	Test test1 = new Test();
	test1.testit();
}
\end{minted}

It also supports a simple arrowlist:

\begin{arrowlist}
\item An example
\item Another example
\end{arrowlist}

And extensive graph drawing using the \texttt{tikz} package:
\begin{figure}
\begin{center}

\begin{tikzpicture}[scale=2]
	% Nodes start here
	\node (A) at (1,0) [circle,draw] {A};
	\node (B) at (1,1.2) [circle,draw] {B};
	\node (C) at (2,1.1) [circle,draw] {C};
	\node (D) at (0.5,1) [circle,draw] {D};
	\node (E) at (2,0) [circle,draw] {E};

	% Lines start here
	\draw[-] (B) to (D);
	\draw[-] (A) to (D);
	\draw[-] (E) to (D);
	\draw[-] (E) to (B);
	\draw[-] (E) to (A);
	\draw[-] (E) to (C);
	\draw[-] (C) to (B);

\end{tikzpicture}
\end{center}
 \caption{Test graph}
\end{figure}

\pagebreak
\section{Page style}
As you can see, this document automatically sets headers according to the current section.

% If you need a bibliography, uncomment this block
% If you want to use this, create a file called "bibliograhpy.bib". We are using normal bibtex for this template.
%\bibliographystyle{IEEEtran}
%\bibliography{bibliography}

\end{document}
